% LaTeX file for resume
% This file uses the resume document class (res.cls)

\documentclass{res}
%\usepackage{helvetica} % uses helvetica postscript font (download helvetica.sty)
%\usepackage{newcent}   % uses new century schoolbook postscript font
\setlength{\textheight}{8in} % increase text height to fit on 1-page
%\usepackage{setspace}
%\singlespacing
%\onehalfspacing
%\usepackage{fullpage}
\usepackage[top=1in, bottom=1in, left=0.5in, right=1.5in]{geometry}

\begin{document}

\name{Jianfeng Jia\\[12pt]}     % the \\[12pt] adds a blank

                        % line after name

%\address{\bf  ADDRESS\\Donald Bren Hall, Room 3019\\University of California,Irvine\\Irvine,CA 92967}
\address{\bf jianfeng.jia@gmail.com (949)678-9893 \\ Seeking a Full-time Big Data Software Engineer Position}

\begin{resume}

\section{EXPERIENCE}
   \vspace{-0.1in}
   \begin{tabbing}
   \hspace{2in}\= \hspace{3in}\= \kill % set up two tab positions
    {\bf Ph.D. Student} \>UC Irvine  \>Sept.2012 - Dec.2017\\
   \end{tabbing}\vspace{-30pt}      % suppress blank line after tabbing
   Big Data Management for Interactive Analytics. \\
   Creator of UCI Cloudberry, committer of Apache AsterixDB, Hyracks, and Pregelix.
   \vspace{-0.1in}
   \begin{tabbing}
   \hspace{2in}\= \hspace{3in}\= \kill % set up two tab positions
    {\bf Software Engineer Intern} \>Sumo Logic, Redwood City,US \>Jun.2015 - Sept.2015\\
   \end{tabbing}\vspace{-30pt}      % suppress blank line after tabbing
   Designed and implemented the lookup operator for a scalable streaming platform.
   \vspace{-0.1in}
   \begin{tabbing}
   \hspace{2in}\= \hspace{3in}\= \kill % set up two tab positions
    {\bf Software Engineer Intern} \>SRCH2.com, Irvine, US     \>Jun.2014 - Sept.2014\\
   \end{tabbing}\vspace{-30pt}      % suppress blank line after tabbing
   Led the Android development group. %Implemented an Instant-Search SDK for Android.
%   \vspace{-0.1in}
%   \begin{tabbing}
%   \hspace{2in}\= \hspace{3in}\= \kill % set up two tab positions
%    {\bf Software Engineer Intern} \>SRCH2.com, Irvine, US     \>Jun.2013 - Sept.2013\\
%   \end{tabbing}\vspace{-30pt}      % suppress blank line after tabbing
%   Implemented a Java SDK for the C++ search engine library. Implemented a Android Search app.
   \vspace{-0.1in}
   \begin{tabbing}
   \hspace{2in}\= \hspace{3in}\= \kill % set up two tab positions
    {\bf Research Engineer} \>Sogou.com, Beijing, China     \>Jul.2008 - Jul.2012\\
   \end{tabbing}\vspace{-30pt}      % suppress blank line after tabbing
   Led a research group of 5 engineers to improve the precision of the Sogou Chinese Input Method product which was used by 300 million people.

\section{EDUCATION}
%    Ph.D. student of Computer Science, University of California,Irvine,from Sept.2012\\
%    M.S. Computer Science, Xiamen University, Xiamen,China, July 2008\\
%        Thesis:\emph{The Application of Dependency Grammar in Chinese-to-English Statistical Machine Translation}\\
%    B.S. Computer Science, Xiamen University, Xiamen,China, July 2005\\
   \vspace{-0.1in}
   \begin{tabbing}
  \hspace{2in}\= \hspace{3in}\= \kill % set up two tab positions
    {\bf Ph.D. Candidate} \>  University of California, Irvine    \>Sept.2012 - Present\\
   \end{tabbing}\vspace{-30pt}      % suppress blank line after tabbing
   Research topic:\emph{Big data management, Large scale data analytics and visualization }
   \vspace{-0.1in}
   \begin{tabbing}
   \hspace{2in}\= \hspace{3in}\= \kill % set up two tab positions
    {\bf M.S. Computer Science} \>Xiamen University, China     \>Sept.2005 - Jul.2008\\
   \end{tabbing}\vspace{-30pt}      % suppress blank line after tabbing
%   Thesis:\emph{The Application of Dependency Grammar in Statistical Machine Translation}
   \vspace{-0.1in}
   \begin{tabbing}
   \hspace{2in}\= \hspace{3in}\= \kill % set up two tab positions
    {\bf B.S. Computer Science} \>Xiamen University, China     \>Sept.2001 - Jul.2005\\
%                             \>Xiamen,China
   \end{tabbing}\vspace{-30pt}      % suppress blank line after tabbing

\section{SKILLS}
    Scala, Java, C++, Android, Python, Hadoop, Spark

\section{SELECTED PROJECTS}
   \vspace{-0.1in}
   \begin{tabbing}
   \hspace{2in}\= \hspace{3in}\= \kill % set up two tab positions
   {\bf Cloudberry System, http://cloudberry.ics.uci.edu}\>  \>UCI\\
   \end{tabbing}\vspace{-30pt}      % suppress blank line after tabbing
   Build a middleware system on top of a parallel database that supports efficient interactive analytics and visualization on billions of records.
   \vspace{-0.1in}
   \begin{tabbing}
   \hspace{2in}\= \hspace{3in}\= \kill % set up two tab positions
    {\bf Lookup Operator for Large-Scale Streaming data}\>  \>Sumo Logic\\
   \end{tabbing}\vspace{-30pt}      % suppress blank line after tabbing
   Developed a lookup operator that can join the large-scale fast streaming data with the static data in databases.
   \vspace{-0.1in}
   \begin{tabbing}
   \hspace{2in}\= \hspace{3in}\= \kill % set up two tab positions
    {\bf Big-Object-Aware Memory Manager in Apache AsterixDB }\>  \>Apache AsterixDB\\
   \end{tabbing}\vspace{-30pt}      % suppress blank line after tabbing
   Designed a big-object-aware memory manager to support the Big-Object feature in AsterixDB.
   %Rewrite the run-time operators(sort, group-by, join, etc) using this memory manager to support the Big-Object feature in AsterixDB.
   \vspace{-0.1in}
   \begin{tabbing}
   \hspace{2in}\= \hspace{3in}\= \kill % set up two tab positions
    {\bf Android SDK for SRCH2 C++ library }\>  \>SRCH2.com\\
   \end{tabbing}\vspace{-30pt}      % suppress blank line after tabbing
   Developed the local and the server version of the Android Search SDK which used the SRCH2 C++ search engine library under the hood. Implemented an error-tolerant search app using this SDK.
   \vspace{-0.1in}
   \begin{tabbing}
   \hspace{2in}\= \hspace{3in}\= \kill % set up two tab positions
    {\bf Big Graph Computing System Evaluation}\> \>UCI     \\
   \end{tabbing}\vspace{-30pt}      % suppress blank line after tabbing
   Built a benchmark to evaluate the performance of popular Graph Computing Systems(Pregelix, GraphX, Giraph, Graph Lab)  . %Evaluated the performance number of Preglix, Giraph, GraphLab, GraphX, Hama.
%
   \vspace{-0.1in}
   \begin{tabbing}
   \hspace{2in}\= \hspace{3in}\= \kill % set up two tab positions
    {\bf Genome Assembling using Hyracks }\> \>UCI     \\
   \end{tabbing}\vspace{-30pt}      % suppress blank line after tabbing
   Used the Hyracks platform to build the genome graph of billions of nodes. Achieved 30\% performance improvement comparing to Hadoop solutions.
%
   \vspace{-0.1in}
   \begin{tabbing}
   \hspace{2in}\= \hspace{3in}\= \kill % set up two tab positions
    {\bf Dashboard Data Flow System using HBase and Pig}\> \>Sogou.com     \\
   \end{tabbing}\vspace{-30pt}      % suppress blank line after tabbing
   Built an data flow processing reporting system that can generate the key metrics of the product every day. 
   The project used HBase to store 15T data (30G per day)  and used Pig scripts to analyze and explore the global user behavior to the online dashboard. %It can also keep track of the individual's statistical input patterns.
%   Through this analyze tool, we found some certain flaw of the product and provided the solution to improve the precision.
   \vspace{-0.1in}
   \begin{tabbing}
   \hspace{2in}\= \hspace{3in}\= \kill % set up two tab positions
    {\bf Large-scale Language Model(LM) for Cloud IME} \> \>Sogou.com     \\
   \end{tabbing}\vspace{-30pt}      % suppress blank line after tabbing
   Built an automatic module for the LM building process updated weekly from the 500G corpus. The system was built on top of Hadoop platform. %Built a decoder using trigram LM and the re-rank model, which precision was 3\% higher than competitors' products.
   \vspace{-0.1in}
   \begin{tabbing}
   \hspace{2in}\= \hspace{3in}\= \kill % set up two tab positions
    {\bf Automatic New Word Detection}\> \>Sogou.com     \\
   \end{tabbing}\vspace{-30pt}      % suppress blank line after tabbing
   Developed a New Word Detection system which was implemented on top of Hadoop platform.\\
%   This system do not need any Linguistics dictionary, fully base on the statistical information of corpus itself.
%   The LM building on those new words improved 1\% precision in desktop IME comparatively to those on normal linguistic words.
   \vspace{-0.1in}
   \begin{tabbing}
   \hspace{2in}\= \hspace{3in}\= \kill % set up two tab positions
    {\bf Dependency Treelet Based Chinese-to-English SMT System} \> \>Xiamen University\\
   \end{tabbing}\vspace{-30pt}      % suppress blank line after tabbing
   Implemented a dependency grammar structure based statistical machine translation system. \\
   \vspace{-0.1in}
%   Model 1 was completely lexicalized; we extract the treelet structure in the source language side and the continuous corresponding string of words in the target language side\\
%   Model 2 applied the generalization to summarize the learned lexical template. Different from the before systems, we applied grammar labels to constrain the generalized template. And the Bleu score of Model 2 was apparent higher than the Model1 and at the same level with the Pharaoh.
%   \begin{tabbing}
%   \hspace{2in}\= \hspace{3in}\= \kill % set up two tab positions
%    {\bf Shift-Reduced Dependency Parser} \> \>Xiamen University\\
%   \end{tabbing}\vspace{-30pt}      % suppress blank line after tabbing
%   Developing an action sequence based deterministic parser.
%   We achieved dependence arc marker accuracy rate (LAS) 76.36\% on Chinese and 82.93\% on the English on the benchmark set in CoNLL2007.


%\section{PATENTS}
%\emph{An Input Method In The Hardware Device}, \textbf{Jianfeng Jia}, Yanfeng Wang, Yang Zhang, CN102087550A public on Jun.8, 2011\\
%\emph{A Method And A System Providing New Words And Hot Words}, \textbf{Jianfeng Jia}, Yang Zhang, Yanfeng Wang, CN102163198A, public on Aug.24, 2011 \\

\section{HONORS AND AWARDS}
%    \begin{itemize}
%     Rhinoceroses prize for building the trigram and re-rank model for cloud IME, Sogou Research, 2010 \\
Google Graduate Student Award in ICS, UC Irvine, 2017 \\
Best Data Visualization Award in Data in UCI Data Science Hackathon, UC Irvine, 2016 \\
Rhinoceroses prize for improving 20\% precision rate in the IME product, Sogou Research, 2009 \\
%     First-Class Scholarship, Xiamen University, 2005 \\
%     First-Class Scholarship, Xiamen University, 2004
%    \end{itemize}

\section{PUBLICATIONS}
%\begin{itemize}
\emph{Drum: A Rhythmic Approach to Interactive Analytics on Large Data}, \textbf{J.Jia}, C.Li, M.Carey.
\emph{Visual Analytics Ecology for Complex System Testing}, S.Su, M.Barton, M.An, V.Perry, C.Li, \textbf{J.Jia}, B.Panneton, Visualization in Practice 2017 at IEEE VIS 2017.
\emph{Caching Geospatial Objects in Web Browsers}, T.Kim, V.Thirumaraiselvan, \textbf{J.Jia}, C.Li, ACM SIGSPATIAL 2017.\\
\emph{Use of Twitter Data to Predict Zika Virus Cases in the United States during the 2016 Epidemic}, S.Masri, \textbf{J.Jia}, C.Li, G.Zhou, M.Lee, G.Yan, J.Wu. PlosONE 2017.\\
\emph{Twitter Coverage of Climate Change and Health before and after the 2016 US Presidential Election}, S.Hopfer, M,Runnerstrom, \textbf{J.Jia}, T.Kim, C.Li. APHA 2017. \\
\emph{Towards Interactive Analytics and Visualization on One Billion Tweets}, \textbf{J.Jia}, C.Li, X.Zhang, C.Li, M.Carey and S.Su, ACM SIGSPATIAL 2016.\\
\emph{Pregelix: Big(ger) Graph Analytics on A Dataflow Engine}, Y.Bu, V.Borkar, \textbf{J.Jia}, M.Carey, T.Condie, VLDB 2015.\\
%\emph{The Application of Statistical Language Model in Sogou Pinyin Input Method Editor}, \textbf{Jianfeng Jia},  Journal of Chinese Association for Artificial Intelligence 2011.vol.1 (4).\\
%     \textbf{Jianfeng Jia}, Xiaodong Shi, Xingbang Lai, \emph{HMM-based Chinese Pinyin Input Method}, J. Modern Computer 2008. (4) 4-6.\\
\emph{Dependency-Based Chinese-English Statistical Machine Translation}, X.Shi, Y.Chen, \textbf{J.Jia}, CICLing 2007.\\
%     \textbf{Jianfeng Jia}, Xiaodong Shi, Yu Chen, \emph{Shift-Reduce Algorithm and Structure Model Based Dependency Statistical Parser}, International Chinese Computing Conference (ICCC) 2007, Wuhan, China.
%\end{itemize}

\end{resume}
\end{document}
